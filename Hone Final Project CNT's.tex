\documentclass[12pt,letterpaper]{beamer}
\usepackage[utf8]{inputenc}
\usepackage{amsmath}
\usepackage{amsfonts}
\usepackage{amssymb}
\usepackage{graphicx}
\usepackage{wrapfig}
\usepackage{caption}
\usepackage[position=b]{subcaption}
\usepackage{float}%put my figures exactly where I want em
\usepackage{grffile}%fix the filenames appearing above images because of spaces in the name
\usepackage[]{microtype}
\DeclareUnicodeCharacter{00A0}{ }

\graphicspath{ {./figures/} }
\title{A Review of Nanoscale Carbon as a Filler in Polymer-Matrix Composites}
\author{Aleks Navratil \\ Hone CNT's Class Fall 2013 \\ Columbia University \\ Department of Mechanical Engineering \vspace{-1.5cm}}
\titlegraphic{Follow along on your mobile device or check references later. Source code, .pdf, and bibliography database are available. Scan here:\\ \includegraphics[scale=.35]{qr}\\
\tiny{https://dl.dropboxusercontent.com/u/28347262/Hone\%20Final\%20Project\%20CNT's.pdf}
}

\date{}
\subtitle[tribo]{For Tribomechanical Systems}

%%This code makes the outline appear at every section break and highlights the current section%%
\AtBeginSection[] { 
  \begin{frame}[plain] 
    \frametitle{Outline} 
    \tableofcontents[currentsection] 
  \end{frame} 
  \addtocounter{framenumber}{-1} 
} 
%%This code does the same thing for subsections. possibly unnecessary%%
\AtBeginSubsection[] { 
  \begin{frame}[plain] 
   \frametitle{Outline} 
    \tableofcontents[currentsubsection] 
    \addtocounter{framenumber}{-1} 
  \end{frame} 
}
\bibliographystyle{plain}

\begin{document}

\begin{frame}[plain]
  \titlepage
\end{frame}

\section{Introduction and Motivation}
\subsection{}

\begin{frame}{Introduction}
\begin{itemize}
\item Development and testing of composite materials is a longstanding research area in tribology

\item Properties of composites:
\begin{itemize}
\item Light weight

\item High strength

\item Low cost

\item Good triboproperties

\begin{itemize}
\item Friction
\item Wear
\item Lubricant wetting
\end{itemize}
\end{itemize}
\end{itemize}
\end{frame}

\begin{frame}{Introduction}
\begin{itemize}
\item Many tribosystems operate in the boundary lubrication regime because of unfriendly conditions
\begin{itemize}
\item Temperature extremes
\item Chemical interactions
\item Severe geometric constraints
\end{itemize}
\item No hydrodynamic lubrication possible under these conditions
\end{itemize}
\end{frame}

\begin{frame}{Introduction}
Composite surfaces are common in many industries:
\begin{itemize}
\item Electronics
\item Aerospace
\item Power generation
\item Automotive
\end{itemize}
\end{frame}

\section{Background}

\begin{frame}{Background}
What are composites?
\begin{itemize}
\item A light matrix material, which dominates composition by volume.
\begin{itemize}
\item PTFE
\item Thermosetting polymer aka epoxies
\item Others
\end{itemize}
\item A high-strength filler material, which is dispersed in the matrix
\begin{itemize}
\item Graphene
\item CNT's
\item Nanodiamonds
\item Others
\end{itemize}
\end{itemize}
\end{frame}

\begin{frame}{Examples of well-known composites}
  \begin{columns}[T]
    \begin{column}{.5\textwidth}
     \begin{block}{}
\begin{itemize}
\item Fibreglass
\item Carbon fiber
\item Reinforced concrete
\item Oriented strand board
\end{itemize}    
    \end{block}
    \end{column}
    \begin{column}{.5\textwidth}
    \begin{block}{}
% Your image included here
\includegraphics[width=\textwidth]{audi}
\vspace{1cm}
\includegraphics[width=\textwidth]{canoe}
    \end{block}
    \end{column}
  \end{columns}
\end{frame}

\begin{frame}{In the old days: Microscale fillers}
\begin{itemize}
\item Microscale fillers in polymer matrices have been shown to reduce wear rates by one or two orders of magnitude. \cite{Lancaster1968549,Tanaka1982221}
\item Polymer tribology is dominated by viscoelasticity and transfer films \cite{bahadur2000development, liu_situ_2012, kragelski1965friction}
\item Possible mechanisms of wear supression:
\begin{itemize}
\item Wear rate limited by strength of composite's transfer film onto counterface \cite{Briscoe197799,ye2012transfer}
\item Presence of fillers reduces average size of wear debris to inhibit wear \cite{Bahadur19841, Blanchet1992229, Ricklin1977487}
\end{itemize}
\end{itemize}
\end{frame}

\begin{frame}{Transfer-film evolution in a PTFE composite}
\begin{figure}
  \centering
    \includegraphics[width=.75\textwidth]{transfer film evolution}
    \caption[Time map of transfer film evolution from PTFE on steel]{A representative time-map of transfer film evolution through the running in, transition, and steady-state wear regimes in linear-sliding PTFE-matrix wear against a steel counterface.}
     \label{fig:transfer_films}
\end{figure}
\end{frame}

\section{Present Research: Nanoscale Fillers}

\begin{frame}{Nanofillers background}
  \begin{columns}[T]
      \begin{column}{.5\textwidth}
    \begin{block}{}
% Your image included here
\includegraphics[width=\textwidth]{cnt}
    \end{block}
    \end{column}
    \begin{column}{.5\textwidth}
     \begin{block}{}
\begin{itemize}
\item Logical continuation of microfiller work from past decades
\item Possibility of improved material properties and performance
\item Carbon nanofillers are particularly promising
\begin{itemize}
\item Graphene
\item CNT's
\item Nanodiamonds
\item Novel fullerenes
\end{itemize}
\end{itemize}   
    \end{block}
    \end{column}
  \end{columns}
\end{frame}

\subsection{Nanofillers in PTFE}

\begin{frame}{PTFE matrix composites}
\begin{itemize}
\item Plenty of research on PTFE matrix composites \cite{mcelwain2008effect, burris2006improved,Burris2009653,chen_tribological_2003}
\item PTFE has low friction but high wear
\item Many efforts to reduce wear via composite loading
\begin{itemize}
\item Carbon fillers
\item Aluminum fillers
\item Nano-silicas and others
\end{itemize}
\item Wear reductions as high as one or two orders of magnitude \cite{MAME200600416,Lai2004916,Blanchet201011,sawyer2003study}
\end{itemize}
\end{frame}

\begin{frame}{PTFE matrix composites}
\begin{itemize}
\item The following images show representative micrographs of graphene's morphology in and out of a PTFE matrix. 
\item Other approaches have also borne fruit, including nanolayered metal-graphene composites
\begin{itemize}
\item Strengths as high as 4.0 GPa
\item Demonstrating graphene's unusually high ability to impede dislocation progagation. \cite{kim2013strengthening}
\end{itemize} 
\end{itemize}
\end{frame}

\begin{frame}{Graphene Composites in pictures}
  \begin{columns}[T]
      \begin{column}{.5\textwidth}
    \begin{block}{}
% Your image included here
        \begin{figure}
                \includegraphics[width=\textwidth]{hi res sem}
                \caption{A high-resolution SEM image showing the microstructure of a graphene-loaded PTFE composite (2\% by mass load fraction.) The inset image shows the rippling edges of graphene platelets.}
                \label{fig:hi_res_sem}
        \end{figure}
        \end{block}
    \end{column}
    \begin{column}{.5\textwidth}
     \begin{block}{}
        \begin{figure}
                \includegraphics[width=\textwidth]{tem graphene on grid}
                \caption{A graphene platelet deposited on an ordinary TEM grid. Note the low opacity of the platelet, even in the non-monolayer regions. The unlabeled scale bar is $.5 \mu$m.}
                \label{fig:graphene_on_grid}
        \end{figure}
    \end{block}
    \end{column}
  \end{columns}
\end{frame}


\subsection{Thermosetting Polymer Matrices}

\begin{frame}{Thermosetting Polymer Matrices}



  \begin{columns}[T]
      \begin{column}{.5\textwidth}
    \begin{block}{}
% Your image included here
        \begin{figure}
                \includegraphics[width=\textwidth]{fracture toughness}
                \caption{Fracture toughness improvements in several composites.}
        \end{figure}
        \end{block}
    \end{column}
    \begin{column}{.5\textwidth}
     \begin{block}{}
\begin{itemize}
\item In epoxy composites, nanoscale fillers including graphene, CNT's, and layered silicates have been shown to improve properties significantly: \cite{blackman2007fracture,Salahuddin20021477,Yasmin20062415}
\begin{itemize}
\item Toughness
\item Elastic modulus
\item Hardness
\end{itemize}
\item Fracture toughness in particular is commonly improved by the introduction of CNT fillers. \cite{Lachman20101093}
\end{itemize} 
    \end{block}
    \end{column}
  \end{columns}
\end{frame}

\begin{frame}{Possible Mechanism of Toughness Improvement}
\begin{figure}[]
  \centering
    \includegraphics[width=.75\textwidth]{cnt crack}
    \caption[Micrograph of a CNT bridging a microcrack in a thermoset]{An SEM micrograph showing a CNT bridging a developing subsurface microcrack in a thermosetting polymer matrix.}
\end{figure}
\end{frame}


\subsection{Other Matrices}

\begin{frame}{Other Matrices}


  \begin{columns}[T]
      \begin{column}{.3\textwidth}
     \begin{block}{}
        \begin{figure}
                \includegraphics[width=\textwidth]{friction plot}
                \caption{Friction is often reduced with increasing filler load.}
        \end{figure}
    \end{block}
    \end{column}  
      \begin{column}{.5\textwidth}
    \begin{block}{}
\begin{itemize}
\item Other polymeric matrices have also been loaded with nanoscale carbon fillers \cite{song2012preparation, li2012preparation}
\item Significant tribological and mechanical improvements have been reported
\begin{itemize}
\item Coefficient-of-friction reductions of more than 90\% at high loading fractions 
\item Reductions in specific wear coefficient of up to an order of magnitude \cite{balazsi2010tribology, pan2012wear,ren2013influence}
\end{itemize} 
\end{itemize}
        \end{block}
    \end{column}
  \end{columns}
\end{frame}

\begin{frame}{Performance Limits}
  \begin{columns}[T]
      \begin{column}{.5\textwidth}
    \begin{block}{}
\begin{itemize}
\item However, friction in monolayer graphene has been shown to depend upon the degree mechanical confinement to the substrate in the system. \cite{PSSB:PSSB201000555}
\item Graphene's frictional behavior has also been shown to depend upon number of layers. \cite{PSSB:PSSB200982329} 
\end{itemize}
        \end{block}
    \end{column}
    \begin{column}{.3\textwidth}
     \begin{block}{}
        \begin{figure}
                \includegraphics[width=\textwidth]{graphene friction}
                \caption{Layer-dependence in graphene CoF.}
        \end{figure}
    \end{block}
    \end{column}
  \end{columns}
\end{frame}

\begin{frame}{AFM Images of Graphene}
\begin{figure}
  \centering
    \includegraphics[width=.9\textwidth]{afm}
    \caption[Tapping-mode AFM images of functionalized graphene]{Atomic force microscope images showing a plan-view of functionalized graphene. (a) is ordinary functionalized graphene, (b) is a proprietary (somewhat thicker) modified graphene. The inset images show cross-sectional height changes.}
\end{figure}
\end{frame}

\begin{frame}{Dispersion Dependence}
\begin{itemize}
\item As dispersion quality rises, property improvements are magnified.
\item However, present dispersion techniques rely on expensive, loud ultrasonication processes
\item Impractical in an industrial setting
\begin{itemize}
\item Cost
\item Operating volume (very loud)
\item Cooling requirements 
\item Exotic solvents, bioincompatible chemistry
\end{itemize} 
\item Further research required to develop new dispersion techniques. Aim for:
\begin{itemize}
\item Same property improvements
\item Reduced processing requirements
\end{itemize}
\end{itemize}
\end{frame}

\section{Graphene Fillers: In Detail}

\begin{frame}{Graphene Fillers in Detail}
  \begin{columns}[T]
      \begin{column}{.5\textwidth}
    \begin{block}{}
\begin{figure}
  \begin{center}
    \includegraphics[width=.9\textwidth]{tensile strength}
  \end{center}
  \caption[Tensile strength improvements for graphene nanocomposites]{Tensile tests and theory}
\end{figure}
        \end{block}
    \end{column}
    \begin{column}{.5\textwidth}
     \begin{block}{}
\begin{itemize}
\item Significant current research interest in graphene fillers
\item May outperform existing nanocomposites
\item Processing costs similar or perhaps reduced 
\end{itemize}
    \end{block}
    \end{column}
  \end{columns}
\end{frame}

\begin{frame}{AFM Images of Graphene}
\begin{figure}
  \centering
    \includegraphics[width=.9\textwidth]{afm}
    \caption[Tapping-mode AFM images of functionalized graphene]{Atomic force microscope images showing a plan-view of functionalized graphene. (a) is ordinary functionalized graphene, (b) is a proprietary (somewhat thicker) modified graphene. The inset images show cross-sectional height changes.}
\end{figure}
\end{frame}

\begin{frame}{Graphene Fillers in Detail}
  \begin{columns}[T]
      \begin{column}{\textwidth}
     \begin{block}{}
\begin{itemize}
\item Graphene's strength and stiffness were characterized by nanoindentation of suspended graphene membranes in 2008
\begin{itemize}
\item Elastic modulus of defect-free samples = 1 TPa
\item Ultimate strength was determined to be 130 GPa at an ultimate strain of nearly 25\% \cite{lee2008measurement} 
\item Corresponding to the intrinsic stress-strain behavior of the carbon-carbon bonds 
\item Strongest material ever measured by man
\end{itemize}
\end{itemize}
    \end{block}      
    \end{column}
  \end{columns}
\end{frame}

\begin{frame}{Early Graphene Experiments}
\begin{figure}
  \begin{center}
    \includegraphics[width=.9\textwidth]{graphene fd}
  \end{center}
  \caption[Tensile strength improvements for graphene nanocomposites]{(A) shows force-displacement curves for graphene. (B) shows stress and deflection against dimensionless distance}
\end{figure}
\end{frame}

\begin{frame}{Graphene vs. Other Fullerenes}
\begin{itemize}
\item Graphene fillers have been shown to outperform carbon nanotube fillers for fracture resistance and other mechanical property improvements by 20-30\% at weight fractions as low as .1\% \cite{njuguna2007epoxy, rafiee_enhanced_2009,rafiee_fracture_2010}
\item Gains attributed to graphene's high specific surface area.
\item Wear reductions of an order of magnitude have been shown in graphene-loaded composites  \cite{kandanur_suppression_2012,shen2014tribological} 
\item Wear reductions accompanied by:
\begin{itemize}
\item Fine wear debris
\item Improved transfer films from highly-loaded samples
\end{itemize}
\end{itemize}
\end{frame}



\begin{frame}{Graphene fillers for other property improvements}
  \begin{columns}[T]
      \begin{column}{.5\textwidth}
    \begin{block}{}
\begin{figure}
  \begin{center}
    \includegraphics[width=\textwidth]{storage modulus}
  \end{center}
  \caption[Tensile strength improvements for graphene nanocomposites]{Viscoelastic onset temperature can be altered by graphene loading and cure parameters}
\end{figure}
        \end{block}
    \end{column}
    \begin{column}{.5\textwidth}
     \begin{block}{}
\begin{itemize}
\item Exotic methods have been found to improve unusual properties of graphene loaded composites
\item Can alter viscoelastic limit onset temperature via UV exposure cure. \cite{MartinGallego20114664}
\end{itemize}
    \end{block}
    \end{column}
  \end{columns}
\end{frame}

\begin{frame}
\begin{itemize}
\item Significant mechanical improvements have been shown in graphene composites 

\item Graphene-oxide fillers at low weight percentages reduce specific wear rates of epoxy materials by around 90\% when compared to neat epoxies \cite{shen_significantly_2013,li2013control} \item Wear behavior of unfunctionalized graphene nanocomposites is not well understood
\item Interference of graphene with debris generation in the epoxy system?
\end{itemize}
\end{frame}

\section{Conclusions}

\begin{frame}{Conclusions}
\begin{itemize}
\item Microscale fillers have long been known to cause measureable triboperformance increases
\begin{itemize}
\item Friction
\item Wear
\item Mechanical Properties
\end{itemize}

\item Recent research has demonstrated that nanoscale fillers can improve properties further still

\item Polymer tribology is dominated by transfer films and viscoelastic effects.

\item Fullerene composite fillers may improve triboproperties by transfer film promotion and interference with debris generation

\item Graphene often outperforms other forms of nanoscale carbon as a filler
\end{itemize}
\end{frame}

\section{References}
\begin{frame}[allowframebreaks]{References}
\bibliography{nanocomposites}
\end{frame}


\end{document}
